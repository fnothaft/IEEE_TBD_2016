\documentclass[10pt,a4paper]{article}
\usepackage[margin=3cm]{geometry}
\usepackage[latin1]{inputenc}
\usepackage{amsmath}
\usepackage{amsfonts}
\usepackage{amssymb}
\usepackage{color}
\usepackage{url}

\newif\ifdraft
\drafttrue
%\draftfalse                                                                              
\ifdraft
\newcommand{\zhaonote}[1]{{\textcolor{green}    { ***Zhao:      #1 }}}
\newcommand{\note}[1]{ {\textcolor{red}    {\bf #1 }}}
\else
\newcommand{\zhaonote}[1]{}
\newcommand{\note}[1]{}
\fi

 \newenvironment{shortlist}{
        \vspace*{-0.5em}
  \begin{itemize}
  \setlength{\itemsep}{-0.1em}
}{
  \end{itemize}
        \vspace*{-0.5em}
}


\title{Scientific Computing Meets Big Data Technology: An Astronomy Use Case}
\author{Zhao Zhang, Kyle Barbary, Frank Austin Nothaft Evan R. Sparks, \\Oliver Zahn, Michael J. Franklin, David A. Patterson, Saul Perlmutter}

\begin{document}
\maketitle
\noindent We are submitting an extended version of the previously published paper~\cite{lamport94} to IEEE Transactions on Big Data.
\subsection*{Summary of Differences}
In this extended manuscript, we migrate the Kira from Scala to Python. 
This iss mainly for better adoption of Kira in the astronomy community. 
We also improve the data structure and layout, and minimize the data copy between language runtimes.
We measure the performance of the latest Kira, and examine how hardware updates, e.g., solid state disks, can further improve the performance.
Enabled by Spark Streaming, we implement and evaluate how the same Kira implementation performed when deployed
as a streaming application. 

\subsection*{New and Original Contributions}
\begin{itemize}
\item{} Big data technology is promising in supporting data-intensive scientific applications.
The latest Kira achieves a $\sim$5x speedup over the HPC solution on Amazon cloud, 
and it is 1.8x as fast as the C implementation on the Edison supercomputer. 
Besides, using solid state disks can boost the performance with an additional 2x speedup.
\item{} With Kira being deployed as a streaming application, we study the tradeoff space between processing latency and throughput of Kira.
A deployment on a 16-node cluster on Amazon cloud can achieve second-scale latency with a sustained throughput of $\sim$600~MB/s.
\end{itemize}

\subsection*{Percentage of New Material}
We update the manuscript to include the performance measurements of the latest Kira. 
Besides, we evaluate how solid state disks can further improve the performance and explored the performance boundary when Kira
is deployed as a streaming application.
With the newly updated 5-page performance evaluation and the according updates throughout the manuscript, we believe there is more
than 30\% new materials compared to the conference version.

\begin{itemize}
\item{} We partially update the abstract, introduction, design and implementation section, and the conclusion to include the latest results.
Specifically, Item 5 in Paragraph 5 of the introduction is newly added.
The third paragraph in Section~2.4 is newly added.
Figure 1 is updated.
Section~4.3 is updated with the streaming interface details.

\item{} Section~8 is completely updated with the measurements of the latest Kira improvements.
Section~8.4 is newly added to include the performance improvements that can be contributed by solid state disks.
Section~8.5 is newly added to present the latency-throughput tradeoff space study when deploying Kira as a streaming application.
\end{itemize}


\begin{thebibliography}{1}

\bibitem{lamport94}
Zhang, Zhao, Kyle Barbary, Frank Austin Nothaft, Evan Sparks, Oliver Zahn, Michael J. Franklin, David A. Patterson, and Saul Perlmutter. "Scientific computing meets big data technology: An astronomy use case." In Big Data (Big Data), 2015 IEEE International Conference on, pp. 918-927. IEEE, 2015.
\end{thebibliography}

\end{document}
