\documentclass[10pt,a4paper]{article}
\usepackage[margin=3cm]{geometry}
\usepackage[latin1]{inputenc}
\usepackage{amsmath}
\usepackage{amsfonts}
\usepackage{amssymb}
\usepackage{color}
\usepackage{url}

\newif\ifdraft
\drafttrue
%\draftfalse                                                                              
\ifdraft
\newcommand{\zhaonote}[1]{{\textcolor{green}    { ***Zhao:      #1 }}}
\newcommand{\note}[1]{ {\textcolor{red}    {\bf #1 }}}
\else
\newcommand{\zhaonote}[1]{}
\newcommand{\note}[1]{}
\fi

 \newenvironment{shortlist}{
        \vspace*{-0.5em}
  \begin{itemize}
  \setlength{\itemsep}{-0.1em}
}{
  \end{itemize}
        \vspace*{-0.5em}
}


\title{Scientific Computing Meets Big Data Technology: An Astronomy Use Case}
\author{Zhao Zhang, Kyle Barbary, Frank Austin Nothaft Evan R. Sparks, \\Oliver Zahn, Michael J. Franklin, David A. Patterson, Saul Perlmutter}

\begin{document}
\maketitle
\noindent We are submitting an extended version of the previously published paper~\cite{lamport94} to IEEE Transactions on Big Data.

In this extended manuscript, we present a completely new implementation of Kira Source Extractor, Kira-SE-v2.
The new implementation shows significant performance improvement, though the key concepts of the Kira system remain the same.
The performance improvement of Kira-SE-v2 fundamentally changes our conclusion in the conference version with 
a 5x and 1.8x speedup over the HPC solution on Amazon cloud and the NERSC Edison supercomputer, respectively.
In addition, we show that Kira-SE-v2 can be 2x faster with solid state disks than spinning disks.
We also study the tradeoff space of latency and throughput of Kira-SE-v2 when it is deployed as a streaming application.
We show that Kira-SE-v2 can achieve second-scale latency with a sustained throughput of $\sim$600~MB/s on a 16-node cluster.

We completely rewrite the performance section (Page 6 -- Page 10) to include the measurements of Kira-SE-v2 and comparison
to the HPC solution on the cloud and the NERSC Edison supercomputer.
Section~8.4 is newly added to include the performance improvements that can be contributed by solid state disks.
Section~8.5 is newly added to present the latency-throughput tradeoff space study when deploying Kira as a streaming application.
The abstract, introduction, design and implementation sections are also updated to reflect the changes in Kira-SE-v2.
In summary, we update more than 50~\% of the conference paper, which is more than the 30~\% new stuff requirement.


\begin{thebibliography}{1}

\bibitem{lamport94}
Zhang, Zhao, Kyle Barbary, Frank Austin Nothaft, Evan Sparks, Oliver Zahn, Michael J. Franklin, David A. Patterson, and Saul Perlmutter. "Scientific computing meets big data technology: An astronomy use case." In Big Data (Big Data), 2015 IEEE International Conference on, pp. 918-927. IEEE, 2015.
\end{thebibliography}

\end{document}
