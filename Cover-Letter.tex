\documentclass[10pt,a4paper]{article}
\usepackage[margin=3cm]{geometry}
\usepackage[latin1]{inputenc}
\usepackage{amsmath}
\usepackage{amsfonts}
\usepackage{amssymb}
\usepackage{color}
\usepackage{url}

\newif\ifdraft
\drafttrue
%\draftfalse                                                                              
\ifdraft
\newcommand{\zhaonote}[1]{{\textcolor{green}    { ***Zhao:      #1 }}}
\newcommand{\note}[1]{ {\textcolor{red}    {\bf #1 }}}
\else
\newcommand{\zhaonote}[1]{}
\newcommand{\note}[1]{}
\fi

 \newenvironment{shortlist}{
        \vspace*{-0.5em}
  \begin{itemize}
  \setlength{\itemsep}{-0.1em}
}{
  \end{itemize}
        \vspace*{-0.5em}
}


\title{Processing Astronomy Imagery Using Big Data Technology}
\author{Zhao Zhang, Kyle Barbary, Frank Austin Nothaft Evan R. Sparks, \\Oliver Zahn, Michael J. Franklin, David A. Patterson, Saul Perlmutter}

\begin{document}
\maketitle
\noindent In response to the invitation from Dr. Qiang Yang on 12/29/2015, we are submitting an extended version of our IEEE Big Data 2015 paper~\cite{zhang2015} for fast track consideration for publication in IEEE Transactions on Big Data.

Previously in our conference paper, we presented Kira, a flexible, scalable, and performant astronomy image processing toolkit using Apache Spark.
We also presented the real world Kira Source Extractor application (Kira SE), and used this application to study the programming flexibility,
dataflow richness, scheduling capacity and performance of the surrounding ecosystem.

The previous Kira SE application was reported as CPU bound, though the original C implementation was disk bound.
From there, we investigated the performance bottleneck and confirmed that the slowdown was due to inefficient communication 
between different programming language runtimes.
Then we optimized Kira SE's data layout and took efforts to make zero-copy calls to external libraries, which lead
to significant performance improvements, though the key concepts of the Kira system remained the same.
In this extended manuscript, we present the completely new implementation of Kira and the Kira Source Extractor application which is referred as Kira-SE-v2.

In particular, we made the following changes:
\begin{itemize}
\item{} The performance improvement of Kira-SE-v2 fundamentally changes our conclusion in the conference version with 
a 5x and 1.8x speedup over the HPC solution on Amazon cloud and the NERSC Edison supercomputer, respectively.
\item{} We also added an investigation of the performance of Kira-SE-v2 using Solid State Disks (SSDs) rather than spinning disks.  
Our results indicate a further speed up of nearly 2x from this change.
\item{} A further addition is an examination of the performance of Kira-SE-v2 when it is implemented as a streaming application.  In this case our results show that Kira-SE-v2 achieves second-scale latency with a sustained throughput of $\sim$600~MB/s on a 16-node cluster.
\end{itemize}

In terms of changes to the text, we completely rewrote and augmented Section 8 (pages 6-10) to include the measurements of Kira-SE-v2 and the comparison
to the HPC solution on the cloud and the NERSC Edison supercomputer.
Section~8.4 was newly added to include the performance improvements that can be contributed by solid state disks.
Section~8.5 was newly added to present the latency-throughput tradeoff study when deploying Kira as a streaming application.
The abstract, introduction, design and implementation sections were also updated to reflect the changes in Kira-SE-v2.
In summary this submission represents a substantial enhancement of our original conference paper.   
We estimate that over 50\% this submission is new and/or updated compared to the conference submission.


\begin{thebibliography}{1}

\bibitem{zhang2015}
Zhang, Zhao, Kyle Barbary, Frank Austin Nothaft, Evan Sparks, Oliver Zahn, Michael J. Franklin, David A. Patterson, and Saul Perlmutter. "Scientific computing meets big data technology: An astronomy use case." In Big Data (Big Data), 2015 IEEE International Conference on, pp. 918-927. IEEE, 2015.
\end{thebibliography}

\end{document}
